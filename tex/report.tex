\documentclass[%
    reprint,
    amsmath,amssymb,
    aps
]{revtex4-2}

\usepackage{graphicx}
\usepackage{bm}

\begin{document}

\preprint{APS/123-QED}

\title{Learning Renormalization Group Flows for Lattices}

\author{Jay Shen}
\thanks{
    My gratitude goes out to Professor Kao at NTU for his guidance, the NTU physics 
    buddies and my group mates for their friendship and hospitality, the other UCTS 
    fellows for their companionship, and Professor Chin and all the UCTS donors 
    for making my summer in Taiwan possible.
}
\affiliation{
    Department of Physics \\
    University of Chicago
}
\email{jshe@uchicago.edu}

\begin{abstract}

    Real space renormalization is a powerful and theoretically fascinating, albeit 
    difficult technique for investigating scale and phase behavior in physical systems. 
    For even the simplest problems, formulating the so-called renormalization group 
    (RG) flow involves incredibly tedious, intuition-dependent work that can drag 
    on for years. 
    However, once accurately described, RG flows have a number of uses, from 
    describing phase behaviors to speeding up simulations. 
    Accordingly, any information at all about their properties is highly valued
    and sought after by physicists. 
    Here, we review and assess a novel approach to real space renormalization 
    initially proposed by \cite[Hou et al.][1]{}. 
    The so-called Machine Learning Renormalization Group (MLRG) algorithm 
    automatically determines approximate RG flows of translationally-invariant 
    Ising models, given only the symmetry description of the lattice. 
    It has the potential to effectively characterize a wide range of 
    interesting systems, and also demonstrates an elegant synthesis of both new 
    and old machine learning techniques with statistical physics. 
    In the first section of the review, we will first give some background for real 
    space renormalization and the Ising model. 
    In the second section, we will describe the MLRG algorithm and demonstrate 
    its use. 
    In the third section, we will discuss the algorithmic design space and suggest 
    modifications for greater efficacy and efficiency. 

\end{abstract}

\maketitle

\section{Background}

\subsection{Real Space Renormalization}

Real space renormalization is a theory of scaling in physical systems. 
That is, it asks how the apparent behavior of a system changes when considered 
from different length scales. 

Consider, as a toy example, the task of modeling some volume of water. 
At the smallest scale, field theories form our understanding of the particulate 
structures within atoms. 
Zooming out and changing our scale of consideration, theories such as quantum 
mechanics and classical mechanics provide well-developed frameworks for modeling 
subatomic and atomic dynamics. 
Zooming out all the way out, fluid mechanics elegantly describes 
the hydrodynamic behaviors of human-scale systems. 

Somehow, all these loosely-related theories must be related, for they 
all model the same system. 
Real space renormalization is a theoretical framework for systematically linking 
these theories. 

\subsection{The Ising Model}

\section{The MLRG Algorithm}

\subsection{Algorithm Description}

\subsection{Demonstration of Results}

\section{The MLRG Design Space}

\subsection{Learning the Flow versus the Monotone}

\subsection{RBM Hyperparameters}

\subsection{Sampler Hyperparameters}

\bibliography{apssamp}

\end{document}